\documentclass[12pt]{scrreprt}
\usepackage[utf8]{inputenc}
\usepackage{blindtext, xcolor}
\usepackage{comment}
\usepackage{enumerate}
\usepackage{booktabs}
\usepackage{multirow}
\usepackage[shortlabels]{enumitem}
\usepackage{graphicx}
\usepackage{tabularx}
\renewcommand\tabularxcolumn[1]{m{#1}}
\usepackage{mathtools}
\usepackage{mathpazo}
\usepackage{mdframed} 
\usepackage{float}
\usepackage{mdwlist}
\usepackage{alphabeta}
\usepackage{makecell}
\usepackage{pdflscape}
\usepackage{geometry}
\usepackage{colortbl}
%---------------------------------------------------------------------------
%--------------------------------------------------------------------------- 
% Full justification with typewriter font 
%--------------------------------------------------------------------------- 
\usepackage{everysel} 
\EverySelectfont{% 
\fontdimen2\font=0.4em% interword space 
\fontdimen3\font=0.2em% interword stretch 
\fontdimen4\font=0.1em% interword shrink 
\fontdimen7\font=0.1em% extra space 
\hyphenchar\font=`\-% to allow hyphenation 
} 
\usepackage[pagewise]{lineno}
%\usepackage[htt]{hyphenat} % Trennung von Typewriter Fonts
\def\linenumberfont{\normalfont\small}
\usepackage{svg}
%\usepackage{glossaries}
\usepackage[acronym,numberedsection]{glossaries}
\usepackage[nohyperlinks, printonlyused, withpage, smaller]{acronym}
\usepackage{listings}
\usepackage{color}
% deutsche Silbentrennung
\usepackage[ngerman]{babel}
% fuer Stichwortverzeichnis 
\usepackage{makeidx}
\usepackage{marvosym}
\usepackage{scrlayer-scrpage}
%\usepackage[headsepline,footsepline,plainfootsepline,markcase=upper]{scrlayer-scrpage}
\usepackage{url}
\usepackage{setspace}
\usepackage{csquotes}

%---------------------------------------------------------------------------
%---------------------------------------------------------------------------
% custom packages go here
\usepackage{xfrac}
%\usepackage{array}
%\usepackage{ragged2e}
\usepackage{boldline} 
\usepackage{array}
\newcolumntype{?}{!{\vrule width 2pt}}
\usepackage{tocloft}
\newcommand{\listequationsname}{}
\newlistof{myequations}{equ}{\listequationsname}
\newcommand{\myequations}[1]{%
\addcontentsline{equ}{myequations}{\protect\numberline{\theequation}#1}\par}
\newcommand\tab[1][3.5cm]{\hspace*{#1}}
%--------------------------------------------------------------------------- 

\usepackage{everysel} 
\onehalfspacing
%adi
\setlength{\headheight}{1.1\baselineskip}
\usepackage[style=apa, backend=biber, language=ngerman]{biblatex}
\DeclareLanguageMapping{ngerman}{ngerman-apa} 
\addbibresource{References.bib} 

%\DeclareLanguageMapping{american}{american-apa}
\bibliography{References}
\renewcommand*{\chapterheadstartvskip}{\vspace*{0cm}}
% http://tex.stackexchange.com/questions/19738/why-doesnt-pagestyleempty-work-on-the-first-page-of-a-chapter
\renewcommand*\chapterpagestyle{scrheadings}
 
%\restylefloat{table}

\definecolor{dkgreen}{rgb}{0,0.6,0}
\definecolor{mauve}{rgb}{0.58,0,0.82}
\definecolor{gray}{rgb}{0.4,0.4,0.4}
\definecolor{darkblue}{rgb}{0.0,0.0,0.6}
\definecolor{cyan}{rgb}{0.0,0.6,0.6}

\lstset{frame=tb,
	captionpos=b,
	language=Java,
	aboveskip=3mm,
	belowskip=3mm,
	showstringspaces=false,
	columns=flexible,
	basicstyle={\small\ttfamily},
	numbers=none,
	numberstyle=\tiny\color{gray},
	keywordstyle=\color{blue},
	commentstyle=\color{dkgreen},
	stringstyle=\color{mauve},
	breaklines=true,
	breakatwhitespace=true,
	tabsize=3
}

\lstdefinelanguage{XML}
{
	morestring=[b]",
	morestring=[s]{>}{<},
	morecomment=[s]{<?}{?>},
	stringstyle=\color{black},
	identifierstyle=\color{darkblue},
	keywordstyle=\color{cyan},
	morekeywords={xmlns,version,type}% list your attributes here
}

% Umbenennung einer Einträge
\renewcommand\tablename{Tabelle}
\renewcommand{\figurename}{Abbildung}
%With babel (and English as language):
%\addto\captionsenglish{\renewcommand{\figurename}{Fig.}}
\renewcommand\bibname{\section{Literaturverzeichnis}}
\renewcommand*\contentsname{Inhaltsverzeichnis}
\renewcommand{\acronymname}{\section{Abkürzungsverzeichnis}}
\renewcommand\lstlistingname{Listing}
\renewcommand\lstlistlistingname{\section{Listingverzeichnis}}

% Damit auch subsubsections nummeriert werden und im ToC auftauchen
\setcounter{secnumdepth}{3}
\setcounter{tocdepth}{2}

\makeindex
\input{sections/Kopf-und-Fusszeile}
%\input{kopf-und-fusszeile2}
\begin{document}
% FOLGENDE ZEILE KANN EINEN FEHLER PRODIZIEREN!
% Die Zeile soll bewirken, dass auf der Titelseite keine
% Seitennummer angegeben wird. Falls das so nicht funktioniert,
% muss ein anderer Workaround herhalten oder die Fußziele in
% zwei Varianten angelegt werden.
\pagenumbering{none}
\begin{titlepage}
	\thispagestyle{scrheadings}
	%Default Layout, clear footer.
	\ofoot{}
%\vspace*{1cm}
%\begin{flushright}
%    \includegraphics{images/FH-burgenland-logo.png}
%\end{flushright}
\noindent
%\vspace*{0.15cm}\text{Fachhochschule Burgenland GmbH}
%\\
%\vspace*{0.15cm}\text{Campus 1}
%\\
%\text{A-7000 Eisenstadt}
\vspace*{1cm}


\begin{center}  % Diplomarbeit ODER Magisterarbeit ODER Dissertation
\usefont{T1}{phv}{b}{n}
	\huge{Aufgabe 2}

    \vspace{3cm}

    \large{
    	Technisches Management von Cloud Solutions\\
        TMCSI
          }
         
        \large{	~\newline \newline
        \\Masterstudiengang Cloud Computing Engineering \\
        SS2023
        }
  
\end{center}
\vspace{1cm}

%\vfill
	\usefont{T1}{phv}{m}{n}
\noindent\begin{tabular}{@{}ll}
Gruppe: & Gruppe A \\
Autoren: & Andreas Gruber, Christian Dragschitz, Dominik Hasiwar \\
Datum: & \today \\
Version: & \docversion
\end{tabular}

\end{titlepage}

\pagenumbering{Roman}
\setcounter{page}{2}
\newpage
\tableofcontents
\clearpage

\pagenumbering{arabic}
\setcounter{page}{1}
% \input{sections/Einleitung}
%	\begin{comment}

%\chapter{Verzeichnisse}
\cleardoublepage
%\appendix
%	\end{comment}
%\chapter{Verzeichnisse}
%http://stackoverflow.com/questions/1243342/how-to-avoid-a-page-break-before-start-of-bibliography
%\nocite{*}
\nocite{back2012web}
% \input{sections/Literaturverzeichnis}

%\bibliography{References}
% \input{sections/Abbildungsverzeichnis}

% \input{sections/Tabellenverzeichnis}
% \input{sections/Formelverzeichnis}

\section{Teilaufgabe 1}

\textit{Beschreibe die Besonderheiten von Release Management im Sinne von SAFe und inwiefern es bei dem Unternehmen aus der Gruppenarbeit anwendbar wäre.}
\\
\\
\ac{safe} ist eine Methode zur Skalierung agiler Prozesse auf Unternehmensebene und kann in Unternehmen jeder Größe und Branche angewendet werden, die komplexe und große Projekte entwickeln. Während sich SCRUM Framework eher an kleinere Teams richtet, wurde \ac{safe} für größere agile Projekte mit mehreren Teams ausgelegt. Das Framework beschreibt sowohl Rollen als auch Zuständigkeiten und definiert strukturierte Leitlinien für die Planung und Verwaltung von Aufgaben. Dabei fördert \ac{safe} die Abstimmung, Zusammenarbeit und Ausführung über zahlreiche agile Teams hinweg. {\cite{ref01}}
\\
\\
Für das Release Management definiert \ac{safe} sogenannte \ac{art}. Dabei handelt es sich um Teams aus mehreren agilen Teams, die zusammen an der Umsetzung eines gemeinsamen Ziels arbeiten. Diese Gruppe von Teams umfasst alle Personen, die notwendig sind und das anvisierte Ziel zu erreichen. Ein \ac{art} besteht in der Regel aus 50 bis maximal 125 Personen und stellt eine virtuelle Organisation innerhalb von \ac{safe} dar, die alle notwendig Arbeiten plant, entwickelt und implementiert. Innerhalb eines \ac{art} werden Arbeiten nach einem festgelegten Zeitplan geliefert und führt. {\cite{ref02}}
\\
\\
Ähnlich zu SCRUM, liefert auch ein \ac{art} in definierten Zyklen ein Inkrement, welchen bereits abgeschlossene Arbeiten enthält. Es wird zwischen einem \ac{si} und einem \ac{pi} unterschieden. Ein \ac{si} ist ein Inkrement, das sich auf die Entwicklung und Integration von funktionsfähigen Systemen oder Produkten konzentriert. Ein \ac{si} ist in der Regel ein zweiwöchiger Zeitrahmen, in dem ein bestimmtes Inkrement oder ein Teil davon entwickelt, getestet und integriert wird. Bei einem \ac{pi} konzentrieren sich die agilen Teams anschließend auf die Entwicklung und Integration mehrerer Systeminkremente. Ein \ac{pi} dauert in der Regel 8-12 Wochen und beinhaltet die Integration von mehreren Systeminkrementen, um ein lauffähiges System oder eine lauffähige Lösung zu schaffen. {\cite{ref03}}
\\
\\
Mit einer Unternehmensgröße von etwas unter 500 Mitarbeitern eignet sich das Bauunternehmen für den Einsatz von \ac{safe}, sofern auch Projekte mit entsprechendem Umfang umgesetzt werden. Wie bei allen agilen Frameworks oder Methoden müssen jedoch auch bei \ac{safe} einige Voraussetzungen erfüllt sein, um eine erfolgreiche Implementierung des Frameworks zu gewährleisten:

\begin{itemize}
    \item Das Unternehmen verfügt über eine klare Geschäftsstrategie und Vision.
    \item Das Management muss die agile Transformation unterstützen und fördern.
    \item Offenheit, Transparenz und Zusammenarbeit.
    \item Das Unternehmen verfügt über ausreichend agile Kompetenzen.
\end{itemize}
\chapter{Teilaufgabe 2}

\textit{Was bedeuten die Abkürzungen MTTF, MTTR, RTO, MTTD, MTTA und MTBF und wo liegen die Unterschiede? Betrachtung soll aus der DevOps Perspektive erfolgen.}
\\
\\
\textbf{\ac{mtbf}} ist die durchschnittliche Zeitspanne zwischen dem Auftreten von System- oder Komponentenausfällen. \ac{mtbf} ist ein wichtiger \ac{kpi}, um sowohl die Verfügbarkeit als auch die Zuverlässigkeit eines Produkts oder Systems zu verfolgen. Je länger die Zeit zwischen Ausfällen ist, desto zuverlässiger ist das System. Das Ziel der meisten Unternehmen ist es, die \ac{mtbf} so hoch wie möglich zu halten. \cite{ref04}
\\
\\
\textbf{\ac{mtta}} \ac{mtta} bezeichnet die durchschnittliche Zeit, die vom Auslösen einer Warnung bis zum Beginn der Bearbeitung des Problems vergeht. Die \ac{mtta}  wird häufig in IT-Service-Management- und Incident-Management-Prozessen verwendet, um die Effektivität der Reaktion eines Teams auf Vorfälle zu messen. \cite{ref05}
\\
\\
\textbf{\ac{mttd}} ist ein Maß für die durchschnittliche Zeitspanne, die zwischen dem Eintreten eines Problems und dessen Erkennung vergeht. Die \ac{mttd} beginnt an dem Punkt, an dem ein Vorfall oder eine Sicherheitsverletzung auftritt, und endet, wenn das Ereignis erkannt und eine Warnung generiert wurde. \cite{ref04}
\\
\\
\textbf{\ac{mttf}} beschreibt die durchschnittliche Betriebsdauer eines Geräts, Produkts oder Systems bis zu dessen Ausfall und liefert somit eine Aussage zur erwartenden Lebensdauer einer Komponente. Die \ac{mttf} wird berechnet, indem die Gesamtbetriebszeit einer Komponente oder eines Systems durch die Anzahl der in dieser Zeit aufgetretenen Ausfälle dividiert wird. Dieser \ac{kpi} wird in der Regel für Komponenten oder Systeme eingesetzt, die bei einem Ausfall  ausgetauscht werden müssen. \cite{ref04}
\\
\\
\textbf{\ac{mttr}} ist die durchschnittliche Zeit, die ein Team dafür aufwenden muss, um ein ausgefallenes System wiederherzustellen. Diese \ac{kpi} umfasst dabei die Zeitdauer, die erforderlich ist, um ein Problem zu erkennen, die Grundursache zu diagnostizieren, die erforderlichen Ersatzteile oder Komponenten zu beschaffen und die Reparatur abzuschließen. \cite{ref06}
\\
\\
\textbf{\ac{rto}} definiert die maximale tolerierbare Zeitspanne, die eine Anwendung, ein System oder ein Prozess ausfallen darf, ohne dass dem betroffenen Unternehmen ein signifikanter Schaden entsteht. Dieser \ac{kpi} ist ein wichtiger Faktor bei der Konzeption und Erstellung eines \ac{drp}. \cite{ref07}

\chapter{Teilaufgabe 3}

\textit{Inwiefern könnte der IaC Ansatz Vorteile bringen, um die zuvor genannten KPI‘s zu verbessern?}
\\
\\
Infrastructure as Code Ist eine Möglichkeit, IT-Infrastruktur als Code darzustellen. Die KPI \textbf{\ac{mttf}} wird verbessert, weil durch die Wiederholbarkeit potentielle Schwachstellen bereits von Anfang an vermieden werden können.
\\
\\
Mit IaC basierte Konfigurationen können schnell Ersatzumgebungen aufgebaut werden.
\textbf{\ac{mttr}} und \textbf{\ac{rto}} können durch automatische Wiederherstellung der Systeme ebenfalls verbessert werden.
\\
\\
In \ac{iac}-Konfigurationen können Monitoring- und Alamierungstools einfach integriert werden, dadurch lassen sich Ausfälle schneller erkennen.
Mittels automatischen Workflows und Alerts werden Probleme schneller behoben. 
Die KPIs \textbf{\ac{mttd}} und \textbf{\ac{mtta}} werden verbessert.
\\
\\
Die KPI \textbf{\ac{mtbf}} kann ebenfalls verbessert werden. Mit \ac{iac}-Scripts können Systeme einfacher aktualisiert und gewartet werden.

\chapter{Teilaufgabe 4}

\textit{Stelle den Unterschied zwischen einem DevOps Engineer und einem Site Reliability Engineer dar. Der Fokus liegt auf den Unterschieden hinsichtlich Aufgaben, benötigtem Knowhow und dem Mehrwert für das Unternehmen.}
\\
\\
\textbf{DevOps Engineer:}
\begin{itemize}
    \item Aufgaben:
          \begin{itemize}
              \item Automatisierung und Integration von Entwicklungsprozessen
              \item Schnelle und zuverlässige Bereitstellung von Software und Services
              \item Zusammenarbeit zwischen Entwicklungs- und Operations-Teams
          \end{itemize}
    \item Knowhow:
          \begin{itemize}
              \item Programmierung, Automatisierung
              \item Systemadministration, Netzwerken und Cloud-Technologien
              \item Kommunikations- und Zusammenarbeitsfähigkeiten
          \end{itemize}
    \item Mehrwert:
          \begin{itemize}
              \item höheren Qualität und Zuverlässigkeit von Software und Services
              \item Software-Entwicklungsprozesse und IT-Infrastruktur verbessern
          \end{itemize}
\end{itemize}

\newpage
\textbf{Reliability Engineer:}
\begin{itemize}
    \item Aufgaben:
          \begin{itemize}
              \item Zuverlässigkeit und Verfügbarkeit von IT-Services
              \item Entwickeln und implementieren Automatisierungstools und Prozesse
              \item Überwachung und Analyse von System- und Anwendungsleistung
          \end{itemize}
    \item Knowhow:
          \begin{itemize}
              \item Systemarchitektur, Netzwerktechnologien
              \item Datenbanken, Skalierbarkeit, Automatisierung und Überwachung
              \item Problembehandlung, Analyse und Troubleshooting
          \end{itemize}
    \item Mehrwert:
          \begin{itemize}
              \item Verfügbarkeit und Zuverlässigkeit von IT-Services und Systemen zu erhöhen
              \item  Reduzierung von Ausfallzeiten und die Verbesserung der Systemleistung
              \item Erhöhung der Kundenzufriedenheit und des Umsatzes
          \end{itemize}
\end{itemize}

DevOps Engineers und Site Reliability Engineers haben unterschiedliche Schwerpunkte, beide tragen aber dazu bei, die IT-Infrastruktur bzw. Software-Entwicklungsprozesse zu optimieren und die Zuverlässigkeit und Verfügbarkeit von IT-Services zu verbessern.
\chapter*{Abkürzungen}

\begin{acronym}
    \acro{art}[ART]{Agile Release Trains}
    \acro{drp}[DRP]{Distaster Recovery Plan}
    \acro{iac}[IaC]{Infrastructure as Code}
    \acro{kpi}[KPI]{Key Performance Indicator}
    \acro{mtbf}[MTBF]{Mean time between failures}
    \acro{mtta}[MTTA]{Mean time to acknowledge}
    \acro{mttd}[MTTD]{Mean time to detect}
    \acro{mttf}[MTTF]{Mean time to failure}
    \acro{mttr}[MTTR]{Mean time to repair}
    \acro{pi}[PI]{Programm Inkrement}
    \acro{pm}[PM]{Produkt Management}
    \acro{rte}[RTE]{Release Train Engineer}
    \acro{rto}[RTO]{Mean time between failures}
    \acro{safe}[SAFe]{Scaled Agile Framework}
    \acro{si}[SI]{System Inkrement}
    \acro{sre}[SRE]{Site Reliability Engineer}
\end{acronym}

\end{document}
