\chapter{Teilaufgabe 3}

\textit{Inwiefern könnte der IaC Ansatz Vorteile bringen, um die zuvor genannten KPI‘s zu verbessern?}
\\
\\
Infrastructure as Code Ist eine Möglichkeit, IT-Infrastruktur als Code darzustellen. Die KPI \textbf{\ac{mttf}} wird verbessert, weil durch die Wiederholbarkeit potentielle Schwachstellen bereits von Anfang an vermieden werden können.
\\
\\
Mit IaC basierte Konfigurationen können schnell Ersatzumgebungen aufgebaut werden.
\textbf{\ac{mttr}} und \textbf{\ac{rto}} können durch automatische Wiederherstellung der Systeme ebenfalls verbessert werden.
\\
\\
In \ac{iac}-Konfigurationen können Monitoring- und Alamierungstools einfach integriert werden, dadurch lassen sich Ausfälle schneller erkennen.
Mittels automatischen Workflows und Alerts werden Probleme schneller behoben. 
Die KPIs \textbf{\ac{mttd}} und \textbf{\ac{mtta}} werden verbessert.
\\
\\
Die KPI \textbf{\ac{mtbf}} kann ebenfalls verbessert werden. Mit \ac{iac}-Scripts können Systeme einfacher aktualisiert und gewartet werden.
