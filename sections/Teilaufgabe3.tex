\chapter{Teilaufgabe 3}

\textit{Inwiefern könnte der IaC Ansatz Vorteile bringen, um die zuvor genannten KPI‘s zu verbessern?}
\\
\\
Infrastructure as Code Ist eine Möglichkeit, IT-Infrastruktur als Code
darzustellen. Die KPI \textbf{MTTF} wird verbessert, weil durch die Wiederholbarkeit
potentielle Schwachstellen bereits von Anfang an vermieden werden können.
\\
\\
Mit IaC basierte Konfigurationen können schnell Ersatzumgebungen aufgebaut werden.
\textbf{MTTR} und \textbf{RTO} können durch automatische Wiederherstellung
der Systeme ebenfalls verbessert werden.
\\
\\
In IaC-Konfigurationen können Monitoring- und Alamierungstools einfach
integriert werden, dadurch lassen sich Ausfälle schneller erkennen.
Mittels automatischen Workflows und Alerts werden Probleme schneller
behoben. Die KPIs \textbf{MTTD} und \textbf{MTTA} werden verbessert.
\\
\\
Die KPI \textbf{MTBF} kann ebenfalls verbessert werden. Mit IaC-Scripts
können Systeme einfacher aktualisiert und gewartet werden.
