\chapter{Teilaufgabe 2}

\textit{Was bedeuten die Abkürzungen MTTF, MTTR, RTO, MTTD, MTTA und MTBF und wo liegen die Unterschiede? Betrachtung soll aus der DevOps Perspektive erfolgen.}
\\
\\
\textbf{\ac{mtbf}} ist die durchschnittliche Zeitspanne zwischen dem Auftreten von System- oder Komponentenausfällen. \ac{mtbf} ist ein wichtiger \ac{kpi}, um sowohl die Verfügbarkeit als auch die Zuverlässigkeit eines Produkts oder Systems zu verfolgen. Je länger die Zeit zwischen Ausfällen ist, desto zuverlässiger ist das System. Das Ziel der meisten Unternehmen ist es, die \ac{mtbf} so hoch wie möglich zu halten. \cite{ref04}
\\
\\
\textbf{\ac{mtta}} \ac{mtta} bezeichnet die durchschnittliche Zeit, die vom Auslösen einer Warnung bis zum Beginn der Bearbeitung des Problems vergeht. Die \ac{mtta}  wird häufig in IT-Service-Management- und Incident-Management-Prozessen verwendet, um die Effektivität der Reaktion eines Teams auf Vorfälle zu messen. \cite{ref05}
\\
\\
\textbf{\ac{mttd}} ist ein Maß für die durchschnittliche Zeitspanne, die zwischen dem Eintreten eines Problems und dessen Erkennung vergeht. Die \ac{mttd} beginnt an dem Punkt, an dem ein Vorfall oder eine Sicherheitsverletzung auftritt, und endet, wenn das Ereignis erkannt und eine Warnung generiert wurde. \cite{ref04}
\\
\\
\textbf{\ac{mttf}} beschreibt die durchschnittliche Betriebsdauer eines Geräts, Produkts oder Systems bis zu dessen Ausfall und liefert somit eine Aussage zur erwartenden Lebensdauer einer Komponente. Die \ac{mttf} wird berechnet, indem die Gesamtbetriebszeit einer Komponente oder eines Systems durch die Anzahl der in dieser Zeit aufgetretenen Ausfälle dividiert wird. Dieser \ac{kpi} wird in der Regel für Komponenten oder Systeme eingesetzt, die bei einem Ausfall  ausgetauscht werden müssen. \cite{ref04}
\\
\\
\textbf{\ac{mttr}} ist die durchschnittliche Zeit, die ein Team dafür aufwenden muss, um ein ausgefallenes System wiederherzustellen. Diese \ac{kpi} umfasst dabei die Zeitdauer, die erforderlich ist, um ein Problem zu erkennen, die Grundursache zu diagnostizieren, die erforderlichen Ersatzteile oder Komponenten zu beschaffen und die Reparatur abzuschließen. \cite{ref06}
\\
\\
\textbf{\ac{rto}} definiert die maximale tolerierbare Zeitspanne, die eine Anwendung, ein System oder ein Prozess ausfallen darf, ohne dass dem betroffenen Unternehmen ein signifikanter Schaden entsteht. Dieser \ac{kpi} ist ein wichtiger Faktor bei der Konzeption und Erstellung eines \ac{drp}. \cite{ref07}
