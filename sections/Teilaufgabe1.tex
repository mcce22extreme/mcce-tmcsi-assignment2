\chapter{Teilaufgabe 1}

\textit{Beschreibe die Besonderheiten von Release Management im Sinne von SAFe und inwiefern es bei dem Unternehmen aus der Gruppenarbeit anwendbar wäre.}
\\
\\
\ac{safe} ist eine Methode zur Skalierung agiler Prozesse auf Unternehmensebene und kann in Unternehmen jeder Größe und Branche angewendet werden, die komplexe und große Projekte entwickeln. Während sich SCRUM Framework eher an kleinere Teams richtet, wurde \ac{safe} für größere agile Projekte mit mehreren Teams ausgelegt. Das Framework beschreibt sowohl Rollen als auch Zuständigkeiten und definiert strukturierte Leitlinien für die Planung und Verwaltung von Aufgaben. Dabei fördert \ac{safe} die Abstimmung, Zusammenarbeit und Ausführung über zahlreiche agile Teams hinweg. {\cite{ref01}}
\\
\\
Für das Release Management definiert \ac{safe} sogenannte \ac{art}. Dabei handelt es sich um Teams aus mehreren agilen Teams, die zusammen an der Umsetzung eines gemeinsamen Ziels arbeiten. Diese Gruppe von Teams umfasst alle Personen, die notwendig sind und das anvisierte Ziel zu erreichen. Ein \ac{art} besteht in der Regel aus 50 bis maximal 125 Personen und stellt eine virtuelle Organisation innerhalb von \ac{safe} dar, die alle notwendig Arbeiten plant, entwickelt und implementiert. {\cite{ref02}}
\\
\\
Ähnlich zu SCRUM, liefert auch ein \ac{art} in definierten Zyklen ein Inkrement, welchen bereits abgeschlossene Arbeiten enthält. Es wird zwischen einem \ac{si} und einem \ac{pi} unterschieden. Ein \ac{si} ist ein Inkrement, das sich auf die Entwicklung und Integration von funktionsfähigen Systemen oder Produkten konzentriert. Ein \ac{si} ist in der Regel ein zweiwöchiger Zeitrahmen, in dem ein bestimmtes Inkrement oder ein Teil davon entwickelt, getestet und integriert wird. Bei einem \ac{pi} konzentrieren sich die agilen Teams anschließend auf die Entwicklung und Integration mehrerer Systeminkremente. Ein \ac{pi} dauert in der Regel 8-12 Wochen und beinhaltet die Integration von mehreren Systeminkrementen, um ein lauffähiges System oder eine lauffähige Lösung zu schaffen. {\cite{ref03}}
\\
\\
Mit einer Unternehmensgröße von etwas unter 500 Mitarbeitern eignet sich das Bauunternehmen für den Einsatz von \ac{safe}, sofern auch Projekte mit entsprechendem Umfang umgesetzt werden. Wie bei allen agilen Frameworks oder Methoden müssen jedoch auch bei \ac{safe} einige Voraussetzungen erfüllt sein, um eine erfolgreiche Implementierung des Frameworks zu gewährleisten:

\begin{itemize}
    \item Das Unternehmen verfügt über eine klare Geschäftsstrategie und Vision.
    \item Das Management muss die agile Transformation unterstützen und fördern.
    \item Offenheit, Transparenz und Zusammenarbeit.
    \item Das Unternehmen verfügt über ausreichend agile Kompetenzen.
\end{itemize}